\documentclass[14p]{book}
\usepackage{marginnote}
\usepackage{color}

\usepackage{tabularx}
\usepackage{verbatimbox}
\definecolor{l}{rgb}{1, 0, 0} %

\begin{document}
\title{Bestiary}
\maketitle

\chapter{Bestiary of the Void}

\emph{This piece of parchment was found near the floating cities of Landan after one of the strongest ethereal storms in recorded history. While language used looks archaic it is still perfectly understandable. A lot of parchment got damaged and can not be read. However, it looks like that the document contains information about mystical fantasy beasts. The fact that monsters can be distinguished to three different and highly separated categories leads to believe that they belong to several different local religions of the past. In here we collect the survived description of these mythical creatures. The occasional references suggesting that author met these creatures in his life means that he was either highly delusional or employed this style to add realism to his work.}

\textbf{Author} The high priest of E...

{\color{l} ---------------------------------------------------------------------------------------------------\\}

\section{First part. Shroom Monsters}
{\color{l} ---------------------------------------------------------------------------------------------------\\}

\textbf{Molögre, The Mating Behemoth}
Many think that tall mushroom forests in the Fields of Shards are dead. Yet they are something entirely different. A more observant viewer would notice that some of the large mushrooms are unlike others. They are a tad shorter, a little darker and always have a fleshy blossom on the tip. I once observed as a huge void bat was devoured by this shroom, which instantly shot up and grabbed the winged beast mid air. But this is neither the most dangerous, neither the most spectacular thing about these forests. Aparently each forest is inhabited by a humongous creature, or more likely, each creature, I named them Molögre, is covered in a mushroom forest. These flesh eating shrooms are mere tentacles of the behemoth. They do contain some sensory organs and may move with tremendous speeds, as I already noted. But more importantly, they are used when two of these creatures meet. I once witnessed the dance of two Molögres. The sound of clashing tentacles can be heard miles away and the view of two large disc like creatures colliding with each other is simply breath taking. From what I have witnessed Molögres are peaceful creatures, if you are smart enough not to get too close to a hungry behemoth. If their tentacles are threatened by any means -- they will first try to hide them in the soil on top of their backs. If that is not possible, or you are a threat to the creature itself -- pray the gods, as there is little you can do against a horde of moose sized mouths which are aiming to eat you.

\emph{One thing still remains a mystery to me: how do Molegroes move? I have not seen any legs, nor they are floating. When they move on surface, their tracks are clearly visible. It looks like a giant rock spitting snail would have leveled whole terrain.}


{\color{l} ---------------------------------------------------------------------------------------------------\\}
\textbf{The Shroom Carrier.}
These creatures looks to be related to moleogres. However, they are much smaller species. The creature has a mushroom-forest-like shell on his back. They roam the Fields on six legs searching for pray. When pray is found a several small lizardish creatures darts out from under their shell and starts ripping the pray to pieces. Being slugish and big this creature relies solely on shroomlings for defence. However, thick armour and mare size usually is enough to stop anyone from trying to eat this creature.

\addvbuffer[12pt 8pt]{%
\begin{tabularx}{\textwidth}{|lX|}
\begin{tabular}[t]{lll}
ST: 10 & HP: 30 & Speed: 3\\
DX: 12 & Will: 20 & Move: 3\\
IQ: 10 & Per: 10 & SM: 10\\
HT: 30 & FP: 10 & Per: 12\\
Dodge: 6 & Parry: 0& DR 5
\end{tabular}
&%
\textbf{Attacks:} Launch  two spawns - 12. 1 dmg on impact. Spawns Shroomling at the tile next to impact.\newline
\textbf{Traits:}Wild Animal \newline
\textbf{Skils:}\newline
\textbf{Class:}\newline
\textbf{Notes:} 
\\
	 \end{tabularx}
}

{\color{l} ---------------------------------------------------------------------------------------------------\\}
\textbf{Shroomling}

They are launched by Shroom carrier. While they are small (size of a rat) and have no armour, they bite is nasty, and the speed is immense.

\addvbuffer[12pt 8pt]{%
\begin{tabularx}{\textwidth}{|lX|}
\begin{tabular}[t]{lll}
ST: 5 & HP: 5 & Speed: 7\\
DX: 12 & Will: 20 & Move: 7\\
IQ: 10 & Per: 10 & SM: 10\\
HT: 5 & FP: 10 & Per: 12\\
Dodge: 6 & Parry: 0& DR 0
\end{tabular}
&%
\textbf{Attacks:} bite 1d-1 imp, claws 1d-3 .\newline
\textbf{Traits:} Can jump from one cell to another if attacked.\newline
\textbf{Skils:}\newline
\textbf{Class:}\newline
\textbf{Notes:} Can not survive alone, needs to be escorted by Shroom carrier.
\\
	 \end{tabularx}
}




{\color{l} ---------------------------------------------------------------------------------------------------\\}

\section{Second part. Hive Mind.}

\textbf{Nah'ra, The Crawling Horror}

A crawling creature that belongs that usually moves in packs of 10 to 20 individuals.

\addvbuffer[12pt 8pt]{%
\begin{tabularx}{\textwidth}{|lX|}
\begin{tabular}[t]{lll}
ST: 8 & HP: 8 & Speed: 4\\
DX: 10 & Will: 20 & Move: 4\\
IQ: 10 & Per: 10 & SM: 10\\
HT: 8 & FP: 10 & Per: 10\\
Dodge: 6 & Parry: 0&
\end{tabular}
&%
\textbf{Attacks:} Jaws-14 1d-1 Cut; Tail-12 1d-3 Cr\newline
\textbf{Traits:}Wild Animal, Mindless \newline
\textbf{Skils:}\newline
\textbf{Class:}\newline
\textbf{Notes:} Really long line full of text that should wrap eventualy. Maybe in next few words? No? But still it must not fall down the page.. Noooooooooo..
\\
	 \end{tabularx}
}

Smallish, two to three feet long, grey scaled maggots with several rows of circular teeth. Nah’ras are brainless eating machines that have sole purpose of devouring anything living or just passed away. They move rather slowly, so a mobile creature may become nah’ras prey only because of it’s own stupidity. Most commonly they move and eat in tight packs forming a frontline of tiny teeth. When attacked they fight as a single group and protect individual members from getting separated. I have seen them to stack into a tight blob, with each maggot facing outwards. They are using this defence against larger and faster predators. Sometimes lone maggots may be encountered, yet I think they are old, sick or simply unlucky enough to have gotten lost from the pack. I have stepped on what looked like dead Nah'ra once. The creature curled around my leg. I blazed the creature as fast as I could, yet the mouth kept on chewing even when most of the body was turning to charcoal. This mistake took a toll of my boot and some of my flesh. From that moment I watch my step here in the VOID.


\emph{BEWARE: While individual creatures are merely a nuisance, the swarm of these creatures can eat down even the largest behemoths.}

{\color{l} ---------------------------------------------------------------------------------------------------\\}

\textbf{Nah'ma} The flying horror. They can fly and desend on their pray from the sky. However, while initial attack can be devastating (attacks with bouth talons), after that they fight on ground. This limits their fighting capabilities to one talon attack per turn since they have to balance on at least three limbs in order not to fall. (engrish is stronk in this one)
 
\addvbuffer[12pt 8pt]{%
\begin{tabularx}{\textwidth}{|lX|}
\begin{tabular}[t]{lll}
ST: 7 & HP: 7 & Speed: 7\\
DX: 10 & Will: 20 & Move: 7\\
IQ: 10 & Per: 10 & SM: 10\\
HT: 8 & FP: 10 & Per: 10\\
Dodge: 8 & Parry: 4&
\end{tabular}
&%
\textbf{Attacks:} Talons-14 1d-1 Inpaling (?)\newline
\textbf{Traits:}Wild Animal, Flyer \newline
\textbf{Skils:}\newline
\textbf{Class:}\newline
\textbf{Notes:} 
\\
	 \end{tabularx}
}

{\color{l} ---------------------------------------------------------------------------------------------------\\}

\textbf{Nah'ka} The pew pew (spiting) horror. Bigger and meaner version of nah’ma. They traded the capability to fly for much more massive body, half of which is covered in bone plates. They hunt in packs of at least three creatures. Approaching to their pray unseen they initiate by short range corrosive spit, that will burn through metal and flesh over course of next few seconds..

\addvbuffer[12pt 8pt]{%
\begin{tabularx}{\textwidth}{|lX|}
\begin{tabular}[t]{lll}
ST: 10 & HP: 12 & Speed: 5\\
DX: 12 & Will: 20 & Move: 5\\
IQ: 10 & Per: 10 & SM: 10\\
HT: 12 & FP: 10 & Per: 12\\
Dodge: 6 & Parry: 0& DR 1
\end{tabular}
&%
\textbf{Attacks:} Spit - 12, corrosive. 1d-3 on hit, h = 1d-1 next turn, h-2, two turns after and so on; Talon - 13, 1d - 1 imp; Tail-12 1d-3 Cr\newline
\textbf{Traits:}Wild Animal, pack hunters \newline
\textbf{Skils:}\newline
\textbf{Class:}\newline
\textbf{Notes:} possible corrosive dmg chains (5 3 1), (4 2), (3 1), (2), (1).
\\
	 \end{tabularx}
}

{\color{l} ---------------------------------------------------------------------------------------------------\\}
\section{Third part. Shapeless.}

\textbf{Shapeless.} I am not entirely sure it is a creature. It moves and kills for sure, but I never saw it eat. It float several feats above ground. For most of the time it looks like a blob of reflective (?) darkness. However, it can change its appearence spontaneously. The new form (there are few predominant ones I have observed) last for about half an hour. The behaviour of the shapeless changes dramatically when it goes through the transformation. While it seems that shapeless can obtaine unlimited amount of forms, some of them are predominant. Here I will list the most common andintriguing appearances I observed.

{\color{l} ---------------------------------------------------------------------------------------------------\\}

\textbf{Shapeless-blades} Shapeless forms a flat disc with razor sharp edges. In this form it is extremely aggressive and must be avoided. I once saw it destroyed a swarm of Nah'ra in just a few minutes.

{\color{l} ---------------------------------------------------------------------------------------------------\\}

\textbf{Shapeless-sheep} One of the most peculiar shapes I observed was a sheep. This creature manifests itself as a black sheep. Creatures behaviour in this form is like a real sheep. Shapeless is calm and spends most of its time grazing shards, roaming around, or even, surprisingly, sleeping. While I have not got a lot of oportunities to observe the shapeless, this was the only non geometrical form. From distance it even apeared to have a soft fur. However, I would not advice trying to touch the creatures.

{\color{l} ---------------------------------------------------------------------------------------------------\\}

\section{Deities}

{\color{l} ---------------------------------------------------------------------------------------------------\\}

\textbf{Endless Wastness} Ever changing and morphing, shining like a sun, mountain size being floats above the landscape. It looks like colorful giant orb surounded by weirdest storms and whirpools i ever seen. Anything touched by the being explodes into piles of sand. Could it be that this being is the passing time representation in the void. Once i saw… \emph{Maybe more observation will be included later} I heard it say once: \emph{Endless Wastness doesn’t care. Endless Wastness does what Endless Wastness does becausse Endless Wastness is Endless Wastness.}

\addvbuffer[12pt 8pt]{%
\begin{tabularx}{\textwidth}{|lX|}
\begin{tabular}[t]{lll}
ST: $\infty$ & HP: $\infty$ & Speed: 10 m/s\\
DX: $\infty$ & Will: $\infty$ & Move: 10 m/s\\
IQ: $\infty$ & Per: $\infty$ & SM: $\infty$\\
HT: $\infty$ & FP: $\infty$ & Per: $\infty$\\
Dodge: $\infty$ & Parry: $\infty$& DR $\infty$
\end{tabular}
&%
\textbf{Attacks:} desintegration touch $\infty$d + $\infty$-1. Turns everything it touches into sand. \newline
\textbf{Traits:} It is a fucking God. \newline
\textbf{Skils:} slowly moves from east to west\newline
\textbf{Class:} \newline
\textbf{Notes:} 
\\
	 \end{tabularx}
}

{\color{l} ---------------------------------------------------------------------------------------------------\\}

\chapter{Bestiary of the Raagna}

\emph{This book is written by $<...>$ in the year $<...>$. It is considered one of the most detailed book describing the beautiful and spectacular creatures of Raagna. The author spend 10 years composing this book, both from his personal experience or told by reliable travellers.}

\section {Desert of $<...>$}

\textbf{The sand wyrm} This colossal beast.

\addvbuffer[12pt 8pt]{%
\begin{tabularx}{\textwidth}{|lX|}
\begin{tabular}[t]{lll}
ST: $\infty$ & HP: $\infty$ & Speed: 10 m/s\\
DX: $\infty$ & Will: $\infty$ & Move: 10 m/s\\
IQ: $\infty$ & Per: $\infty$ & SM: $\infty$\\
HT: $\infty$ & FP: $\infty$ & Per: $\infty$\\
Dodge: $\infty$ & Parry: $\infty$& DR $\infty$
\end{tabular}
&%
\textbf{Attacks:} desintegration touch $\infty$d + $\infty$-1. Turns everything it touches into sand. \newline
\textbf{Traits:} It is a fucking God. \newline
\textbf{Skils:} slowly moves from east to west\newline
\textbf{Class:} \newline
\textbf{Notes:} 
\\
	 \end{tabularx}
}


\end{document}





